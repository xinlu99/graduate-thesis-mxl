%%
% BIThesis 研究生学位论文模板 The BIThesis Template for Graduate Thesis
% This file has no copyright assigned and is placed in the Public Domain.
%%

\chapter{绪论}

\label{chap:intro}

\section{研究目的与意义}

随着“网络强国”和“航天强国”战略的深入实施以及全球数字化转型的加速推进,天地一体化信息网络(Space-Air-Ground Integrated Network, SAGIN)作为国家关键信息基础设施,其战略地位日益凸显。天基网络,特别是以低轨道(Low Earth Orbit, LEO)卫星星座为骨干的系统,正由传统的通信中继与数据传输管道,演进为集通信、计算、感知于一体的“天基云网”。这一演进使得卫星网络不仅能够为海洋、沙漠、山区等地面蜂窝网络难以覆盖的区域提供无缝宽带接入服务,还开始承载日益复杂的在轨实时处理任务,如对地观测数据智能分析、空间环境监测、应急通信保障以及各类在轨科学试验等。

在此背景下,卫星终端作为天基网络的“神经末梢”,承担着连接物理世界与数字空间的前哨角色。在各类在轨试验与探测任务中,卫星终端负责海量、多源、异构数据的感知、采集与初步处理。然而,随着任务复杂度不断提升,试验业务呈现出典型的计算密集型特征。例如,高分辨率遥感图像在轨识别、空间态势感知实时分析、在轨航天器故障快速诊断等任务,对终端的即时计算能力提出了极高要求。同时,卫星终端作为特殊嵌入式设备,其设计受到体积(Size)、重量(Weight)、功耗(Power)与成本(Cost)的“SWaP-C”约束;星载处理器还需经过抗辐射加固设计,这进一步限制了其计算性能的提升。因此,卫星终端普遍面临计算、存储与能量资源极其有限的现实困境。

当大量计算密集型试验任务并发产生时,仅依靠单个卫星终端的本地计算能力已难以为继。为此,计算卸载(Computation Offloading)技术应运而生,成为提升卫星终端任务处理能力的关键使能技术。其核心思想是将终端难以高效处理的计算任务,通过星间链路或星地链路迁移至计算资源更为充裕的节点(如协同卫星、地面站或云数据中心)执行,再将结果返回任务发起终端。

早期计算卸载方案大多采用“星-地”模式,将海量原始数据回传至地面数据中心集中处理,形成典型的“空采地算”范式。该模式虽然能够充分利用地面强大的计算能力,但也暴露出突出问题:一是星地链路带宽有限且可见时间窗口短暂,海量数据回传成为系统性能瓶颈;二是星地往返路径长,往返时延(Round-Trip Time, RTT)难以满足应急响应、实时控制等时延敏感型任务需求;三是高度集中的决策模式使地面中心站成为性能与可靠性瓶颈,一旦故障可能导致大范围服务中断。

为克服上述问题,卫星边缘计算(Satellite Edge Computing, SEC)或天基边缘计算(Space-based Edge Computing)作为新兴计算范式受到广泛关注。其核心理念是将计算与存储能力向网络边缘(即卫星终端或星座边缘节点)下沉,构建分布式计算平台。在该架构下,资源受限的卫星终端可将任务卸载至邻近、计算能力更强的卫星边缘节点,实现“数据在轨处理、结果回传地面”的高效模式。这不仅大幅减轻星地链路负担、显著降低任务响应时延,还通过分布式协同增强天基网络整体的韧性与可扩展性。

然而,将地面边缘计算技术直接迁移至天基网络仍面临一系列独特挑战:

(1)分布式环境下的信任缺失与协同困境:天基网络由分属不同运营方、覆盖不同区域的卫星节点构成,是典型的多主体分布式系统。在缺乏中心化可信第三方的情况下,如何保障试验任务从规划、分发、执行到结果回收等全流程的公开透明与不可篡改,成为多方可信协作的首要难题。试验数据每一次流转与处理,都需要可靠的存证与溯源机制,以防止数据在传输和计算过程中被恶意篡改或伪造,确保最终试验结果的公信力。

(2)高并发任务下的并发开销与资源竞争:在真实试验场景中,多个卫星终端可能同时向同一边缘节点卸载任务。新任务的到来会与正在执行的任务争夺CPU周期、内存等有限资源,使所有任务的完成时间普遍延长。这种由资源竞争引发的额外性能损耗即“并发开销”,却常被传统卸载策略忽略。如何准确建模与量化并发开销,使卸载决策既考虑新任务自身时延,又兼顾其对系统中存量任务性能的影响,是实现全局最优的关键。

(3)网络拓扑高动态与资源状态不确定性:尤其在LEO星座中,卫星高速运动、星间链路频繁切换,网络拓扑结构时刻变化,导致节点难以实时获取精确的全局状态信息(如各节点负载、链路质量等)。依赖“全局信息透明”假设的集中式优化算法在此环境下不仅信令开销巨大,且信息滞后严重,难以适应快速变化的网络。亟需在信息不完全、时变动态的分布式环境下,构建高效、鲁棒的动态卸载决策与资源调度智能方法。

针对上述挑战,本文拟探索将区块链(Blockchain)、联邦学习(Federated Learning, FL)与深度强化学习(Deep Reinforcement Learning, DRL)等前沿技术深度融合,构建一套面向卫星终端试验任务的分布式、智能化数据管理与计算卸载框架。具体而言,利用区块链的去中心化、不可篡改与可追溯特性,为多主体协作构建可信基础,实现试验任务全生命周期的可信存证与协同管理;在此基础上,引入联邦学习机制,在不泄露各终端原始数据的前提下,协同构建高精度的并发开销预测模型,使终端能够更精准地感知全局负载状态;最终,借助深度强化学习,使各终端智能体在与高动态环境的持续交互中,自主学习最优计算卸载与资源调度策略,实现时延、能耗与负载均衡等多目标的动态优化。

因此,本文的核心目的与意义在于:面向天基网络中卫星终端资源受限、任务并发和网络高动态等现实瓶颈,系统性解决分布式环境下试验任务数据可信管理难、计算资源协同调度效率低的关键问题。通过构建一个可信、智能、高效的计算卸载架构,不仅能够显著提升卫星终端任务处理效能与系统吞吐量,保障试验数据安全与隐私,还可为我国构建自主可控、智能泛在的天地一体化信息网络提供新的理论方法与关键技术支撑,具有重要的理论研究价值和广阔的工程应用前景。

\section{国内外研究现状与发展趋势}

本节围绕“基于区块链的卫星终端试验任务数据管理与计算卸载”这一核心议题,从天基网络计算范式演进、区块链赋能的可信协同、联邦学习驱动的协作建模以及深度强化学习引导的动态资源优化四个维度,对国内外前沿研究现状进行系统综述,梳理核心进展,分析现有技术局限,并据此明确本文的研究切入点与创新价值。

\subsection{天基网络架构与卫星边缘计算}

随着“新基建”和全球数字化进程的深入推进,以LEO卫星星座为核心的天地一体化信息网络(SAGIN)正由传统通信管道向集通信、计算、感知于一体的“天基云网”演进。为缓解海量数据回传瓶颈、满足时延敏感任务需求,卫星边缘计算(Satellite Edge Computing, SEC)已成为重要研究方向,其核心思想是将计算能力由地面中心前移至轨道边缘,实现数据的在轨智能处理。

近年来研究主要聚焦于SEC的架构设计、资源管理与应用赋能。在顶层架构方面,大量综述性工作为该领域描绘了整体蓝图。Cheng等在文献\cite{Cheng2022}中系统阐述了面向6G的SAGIN服务化网络架构,强调异构资源协同与云边融合的重要性;Zhou等在\cite{Zhou2023}中则从航空航天一体化网络视角,深入探讨了网络动态性建模、理论性能分析与系统优化等关键问题,并指出超密集卫星星座带来的新挑战。这些工作共同表明:将云计算与边缘计算能力从地面延伸至太空,构建分布式、智能化天基计算体系是必然趋势。

在关键技术挑战方面,Wang和Li在\cite{Wang2023}中明确指出卫星计算面临资源受限、能量约束与拓扑高动态等核心问题,并以“天算星座”为例展示了开放卫星研究平台的实践探索。这些挑战推动了面向SEC场景的计算卸载与资源分配算法研究。早期工作多采用集中式优化或传统博弈论方法,但普遍难以适应天基网络的强动态性和不确定性。

总体来看,现有研究的主要局限体现在:

(1)对网络高度动态性的适应不足:多数研究基于简化网络模型,未充分考虑卫星高速移动与星间链路频繁切换对卸载连续性和策略稳定性的冲击。

(2)对“并发开销”的忽视:传统模型往往假设边缘节点资源可被线性划分,忽略多任务并发执行时因缓存、内存带宽等资源竞争导致的非线性性能下降,使时延和能耗估计与实际偏差较大。

(3)依赖理想化全局信息:许多算法假设能够实时获取精确的全局网络状态,这在分布式、大时延的天基网络中不仅信令开销巨大,而且难以实现。

因此,面向真实天基环境,设计能够在不完全信息下适应高动态拓扑、并精准刻画并发性能的分布式智能卸载与资源管理方案,已成为亟待突破的关键问题,也构成了本文的研究起点。

\subsection{区块链赋能的分布式可信协同与数据管理}

在多运营主体共存的分布式天基网络中,构建互信机制并保障数据全生命周期的完整性与可追溯性,是实现高效协同的前提。区块链技术以去中心化、防篡改与可追溯等特性,为解决该问题提供了新的技术路径。

当前,将区块链与边缘计算相结合的研究已成为热点,尤其在与天基网络具有类似分布式特征的车联网(Vehicular Networks)中。Nguyen等在\cite{Nguyen2023}中提出了一种区块链支持的边缘计算协作卸载与区块挖掘方案,采用多智能体DRL联合优化卸载决策与资源分配;Fan等在\cite{Fan2024}中面向云–边–端协同网络,研究了区块链保障下的任务卸载与资源分配,通过Lyapunov优化与DRL结合,最小化任务时延与能耗;Wang等在\cite{Wang2023b}中则探讨了区块链支撑的数字孪生物联网可持续计算资源管理问题。

这些研究表明,区块链在分布式边缘环境中保障任务安全、构建激励机制与促进资源共享方面具有巨大潜力,核心思想可迁移至天基网络场景,主要体现在:

(1)任务全生命周期可信存证:将卫星试验任务的发布参数、执行节点、起止时间、结果哈希等关键信息作为交易写入区块链,形成不可篡改的“分布式任务日志”,支撑任务审计、故障排查与多方协作。

(2)非实时全局状态共享:通过读取链上数据,授权卫星终端可获知其他节点的任务执行概况,在无需实时点对点通信的情况下形成准实时、一致性全局状态视图,为分布式决策提供支撑。

(3)基于智能合约的自动化协同治理:通过智能合约定义任务分配规则、资源访问控制与奖惩机制,实现去中心化、自动化的资源调度与治理。

然而,现有工作大多面向地面边缘网络,直接应用于卫星场景仍面临两方面挑战:其一,传统共识机制(如PoW)计算与能耗开销巨大,难以适应星上资源极度受限的环境;其二,将全部数据直接上链既不现实也不高效,亟需设计高效的链上链下协同机制。

本文工作与上述研究密切相关,但更加聚焦天基网络的特殊约束:拟采用轻量级联盟链共识机制(如PoA及其变体),并设计分层链架构以解耦控制信令与大规模试验数据,实现非实时全局状态共享与可信协同的区块链底座。这构成了本文的第一项核心贡献。

\subsection{联邦学习与隐私保护的协作式建模}

在分布式智能卸载决策中,单个终端如何准确感知与预测全网状态,尤其是其他节点的资源负载,是一项关键难题。联邦学习(Federated Learning, FL)作为“数据不动、模型移动”的分布式学习范式,为此提供了理想的隐私保护框架。

FL在边缘网络中的研究发展迅速。Zhai等在\cite{Zhai2024}中提出了专为LEO卫星网络设计的去中心化联邦学习框架FedLEO,通过计算卸载辅助模型训练,有效应对卫星间歇性连接和资源异构性问题。面向车载网络场景,Zhao等在\cite{Zhao2024}中将FL与多智能体DRL结合,在保护数据隐私的同时,通过协作训练提升任务卸载与资源分配策略的性能,并缓解数据非独立同分布(Non-IID)问题。

这些研究启发我们,将FL应用于解决计算卸载中长期被忽视的“并发开销”预测问题。并发开销的准确建模是DRL智能体能否做出近似最优卸载决策的前提。然而,单个卫星节点历史运行数据有限,难以训练出具有良好泛化能力的模型;同时,直接共享包含任务细节与性能指标的原始数据又带来严重隐私风险。

现有FL相关研究多聚焦于DRL策略本身的联邦训练,或一般资源预测问题,鲜有工作针对并发执行引起的非线性性能影响进行专门建模。考虑到并发开销受硬件微结构、操作系统调度及任务类型组合等多因素影响,难以用简单数学公式刻画,典型适合采用数据驱动方法。

为弥补这一空白,本文提出利用联邦学习协同训练并发开销预测模型。具体而言,每个卫星边缘节点基于本地历史任务运行数据,训练一个深度神经网络(如卷积神经网络,CNN)以拟合其在不同负载与任务组合下的性能表现;随后通过联邦学习(如FedAvg)对各节点本地模型参数进行安全聚合,得到精度更高、泛化能力更强的全局并发开销预测模型。终端在进行卸载决策时,可基于该模型输入任务信息和目标节点排队情况,获得更精准的任务完成时延预测,从而有效提升决策质量。

这构成本文的第二项核心贡献:通过隐私保护的协作建模,解决智能卸载中关键环境参数(并发开销)的感知难题,避免盲目卸载引发的“雪崩效应”,为资源竞争场景下的性能预测提供新思路。

\subsection{深度强化学习驱动的动态任务卸载与资源优化}

在拓扑高动态、链路不稳定及资源状态时变的天基网络中,传统依赖精确模型的优化方法往往难以奏效。深度强化学习(DRL)作为一种无模型决策优化技术,通过智能体与环境的持续交互与试错学习,可在复杂约束下自主寻优,已成为解决动态卸载与资源分配问题的重要工具。

DRL在边缘计算任务卸载中的应用已较为广泛。Yamansavascilar等在\cite{Yamansavascilar2023}中提出DeepEdge框架,利用双重深度Q网络(DDQN)处理动态卸载请求与时变信道,以最大化任务完成率;Xie和Cui在\cite{Xie2025}中则针对排队时延导致奖励信息失真问题,提出改进的PPO算法以提升策略性能与收敛稳定性。

在卫星网络领域,Zhang等在\cite{Zhang2024}中围绕能效优化,设计多跳对等卸载方案,并结合Lyapunov框架与在线学习提升能效与时延表现;Zhang等在\cite{Zhang2023}中则采用深度确定性策略梯度(DDPG)算法,实现对连续与离散混合动作空间(卸载决策与资源分配比例)的协同优化。进一步地,Zhao等在\cite{Zhao2025}中提出基于时空注意力机制的PPO算法(STA-PPO),分别捕捉任务到达的时间依赖性与卫星运动的空间动态,以应对LEO网络独特的时空特性;Chen等在\cite{Chen2024}中提出SpaceEdge方案,通过DRL优化服务迁移与功率控制,提升服务连续性与可持续性;Han等在\cite{Han2023}中则采用两时间尺度学习框架,解耦长期资源分配与短期卸载决策,有效降低决策复杂度。

尽管上述工作在动态决策优化方面取得显著进展,但仍存在共性不足:

(1)环境状态感知不完整:大多数DRL智能体的状态构造依赖理想化局部信息,缺乏对全网负载与任务竞争情况的可靠、一致视图,易导致策略次优。

(2)决策成本预估不准确:对并发开销的简化建模甚至忽略,使智能体在评估卸载动作价值(Q值)时所依赖的成本函数与真实环境存在偏差,不利于策略收敛与性能发挥。

(3)多目标优化权衡不足:通常通过固定加权和构造奖励函数,难以灵活适配不同业务场景下对时延、能耗与负载均衡等指标优先级的动态变化。

本文工作针对上述不足进行了有针对性的改进:首先,依托区块链实现的非实时全局状态共享机制,为DRL智能体提供更全面、可信的状态输入;其次,通过联邦学习训练的并发开销预测模型,为奖励函数提供更准确的成本评估。在此基础上,将计算卸载问题建模为马尔可夫决策过程(MDP),并采用改进的DQN或PPO算法进行求解,使智能体在更接近真实的环境中学习,获得兼顾时延、能耗与负载均衡的鲁棒高效策略。这构成了本文的第三项核心贡献,形成“区块链提供状态—FL预测成本—DRL优化决策”的技术闭环。

\section{本论文研究内容与主要贡献}

面向天基网络环境下卫星终端资源受限、试验任务并发、协同信任缺失以及网络高动态等核心挑战,本文以构建高效、智能、可信的卫星终端试验任务数据管理与计算卸载系统为总体目标,提出一种深度融合区块链、联邦学习与深度强化学习的创新架构。本文主要研究内容与贡献概括如下。

(1)基于区块链的卫星试验任务可信存证与协同管理。

为解决分布式多主体天基网络中的信任缺失问题,保障试验任务全生命周期的安全、透明与一致性,本文首先设计了一种基于联盟链的分布式数据管理与协同框架。构建面向天基试验任务的分层式联盟链架构,包括负责全局任务协调与共识的主链,以及负责具体试验数据存证的多个子链,并通过跨链机制实现主子链间的信息交互,以分离控制信令与海量试验数据,提升系统吞吐量与可扩展性。利用智能合约对试验任务的发布、接收、执行、结果验证等关键流程进行编码与自动化,实现任务规则、参与方权责及奖惩机制的去中心化执行与不可篡改。进一步,设计非实时全局状态共享机制,将任务卸载事件(如任务开始和结束)以交易形式记录在链上,使任一终端均可通过链上查询获取系统中任务执行状态,为后续智能卸载决策提供可信、一致的全局视图。

(2)基于联邦学习的并发开销预测与卸载决策优化。

针对传统卸载策略普遍忽略多任务并发执行导致的“并发开销”问题,本文创新性引入联邦学习,在不泄露各节点原始运行数据前提下,协同构建高精度并发开销预测模型,以优化卸载决策。首先分析并形式化定义任务并发开销,将其刻画为新任务抵达后因资源竞争对自身及在执行任务造成的额外时延,并建模为复杂非线性函数预测问题。进而提出基于联邦学习与深度Q网络(FL-DQN)的分布式卸载策略:采用卷积神经网络(CNN)拟合并发开销函数,各卫星边缘节点利用本地历史任务执行日志进行本地训练,再通过FL机制(如FedAvg)在不交换原始数据的情况下聚合模型参数,得到精度更高、泛化更好的全局并发开销预测模型。该模型分发至所有终端,用于卸载决策时精准预估任务真实完成时延。

(3)基于深度强化学习的动态资源调度与节点优选。

为适应拓扑高动态、链路不稳定与资源状态不确定的天基网络环境,本文采用深度强化学习方法设计动态资源调度与最优卸载目标选择策略。将单个卫星终端的计算卸载过程建模为马尔可夫决策过程(MDP),其中状态空间包括终端任务信息、通过区块链获取的邻近节点负载信息以及链路质量等;动作空间为决策选项(本地执行或卸载至某可用邻近卫星节点);奖励函数综合考虑任务完成时延、终端能耗和网络负载均衡度等指标。在此基础上,采用改进的PPO或DQN算法训练DRL智能体,通过与模拟环境的持续交互,在无精确系统模型的情况下逐步学习最大化长期累积奖励的近似最优策略,实现面向新到达任务的动态执行方式与卸载目标选择。

(4)天基分布式试验智能管控平台的设计与验证。

为验证所提理论方法的有效性与工程可行性,本文基于前述架构与算法,设计并实现了一个系统级演示与验证平台。平台集成四大核心模块:卫星终端与网络环境模拟器,用于生成多类型试验任务、模拟卫星拓扑动态变化与星间链路特性;分布式区块链网络,实现分层联盟链与智能合约部署,负责任务可信管理与状态共享;智能调度引擎,内嵌基于FL训练的并发开销预测模型与基于DRL的动态卸载与资源调度算法;可视化监控与分析系统,用于实时展示系统运行状态、任务流转过程及各类性能指标。通过与本地执行、随机卸载、贪心卸载等基准算法的对比实验,验证本方案在降低任务平均完成时延、提升系统吞吐量、保障数据安全与隐私及实现网络负载均衡等方面的优势。

\section{论文结构安排}

本文共分为六章,主要安排如下:

第一章为绪论,介绍研究背景与意义,分析天地一体化信息网络发展趋势及卫星终端在试验任务场景下面临的资源受限与计算卸载挑战,综述天基网络计算卸载、区块链数据管理、联邦学习与深度强化学习等相关领域研究现状,明确本文研究内容、创新点与研究目标,并给出全文结构安排。

第二章为相关理论与关键技术,系统介绍本文所涉及的基础理论与关键技术,包括计算卸载的基本概念与模型;区块链技术原理,重点为联盟链、智能合约与共识机制;联邦学习框架与工作流程及其在隐私保护方面的优势;深度强化学习的基本要素(如MDP),以及DQN、PPO等典型算法。

第三章为基于区块链的试验任务可信管理架构,详细阐述本文提出的任务管理与数据存证方案。设计面向天基试验场景的分层联盟链架构,定义主链与子链的功能定位与交互方式;设计基于智能合约的任务全生命周期管理流程,并给出区块结构与数据格式;说明如何利用该区块链架构实现分布式环境下的任务状态共享与协同。

第四章为基于联邦学习与深度强化学习的计算卸载策略,是全文核心章节。重点介绍智能计算卸载算法:构建系统模型,对任务、通信、计算及“并发开销”进行数学形式化定义;详细说明基于联邦学习的并发开销预测模型设计与训练过程;将计算卸载问题建模为MDP,设计基于DRL的动态卸载与资源调度算法;给出FL–DRL协同工作的完整流程。

第五章为仿真实验与性能评估,对所提方案进行系统级评估。介绍仿真平台参数设置、数据集构建与对比基准算法选取;通过一系列实验,从任务平均完成时延、系统吞吐量、终端能耗、网络负载均衡度等多个维度评估所提方案性能,并对实验结果进行深入分析与讨论。

第六章为总结与展望,对全文研究工作进行总结,重申主要贡献与创新点,分析当前研究的不足,并展望未来可进一步研究的方向,如更精细的移动性管理、更鲁棒的异构联邦学习算法以及多智能体协同策略的演进等。
