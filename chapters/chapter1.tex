%%
% BIThesis 研究生学位论文模板 The BIThesis Template for Graduate Thesis
% This file has no copyright assigned and is placed in the Public Domain.
%%

\chapter{绪论}

\textcolor{blue}{
  正文包括绪论、论文具体研究内容及结论部分。博士学位论文:一般为6~10万字,其中绪论要求为1万字左右。硕士学位论文:一般为3~5万字,其中绪论要求为0.5万字左右。(外语学科:中文、日文不少于3万字,西文2万字左右。)
}

\textcolor{blue}{
  绪论一般作为第1章。绪论应包括本研究课题的学术背景及其理论与实际意义;本领域的国内外研究进展及成果、存在的不足或有待深入研究的问题;本研究课题的来源及主要研究内容等。
}


\label{chap:intro}
\section{本论文研究的目的和意义}

随着航天技术的飞速发展及"新基建"战略的深入推进,天基信息网络(Space-based Information Network)作为国家战略性基础设施,正在经历从传统的"天星地网"向"天基云网"的跨越式转变。天基网络利用低轨卫星(LEO)星座,为地面网络无法覆盖的偏远地区、远洋航行及航空运输等场景提供无缝、高效的全球通信与数据服务。随着卫星互联网建设的加速,卫星节点及试验终端的数量呈指数级增长,其承载的业务类型也从简单的数据透传向复杂的遥感数据处理、即时试验任务分析以及边缘智能应用拓展。天基网络中产生的海量数据对系统的任务调度能力、数据处理效率及信息可信管理提出了前所未有的挑战。早期的一些研究方案中,卫星通常作为"弯管"节点,依赖地面测控站或云数据中心,将海量原始数据回传至地面进行处理。然而,受限于星地链路的有限带宽、可见时间窗口短暂以及长传播时延等物理约束,这种"空迈地算"的集中式处理模式已难以满足日益增长的时延敏感型任务(如实时故障诊断、应急响应试验)的需求。

近年来,得益于星载计算能力的提升,卫星边缘计算(Satellite Edge Computing)作为一种更贴近数据源头的处理范式,获得了科研人员的广泛关注。其功能上与地面边缘计算类似,通过赋予卫星节点一定的计算与存储能力,使其能够就近处理试验任务与业务数据。这种架构不仅有效减轻了星地回传链路的压力,显著缩短了任务响应时间,还在一定程度上提升了天基系统的服务质量与抗毁顽存能力。在该架构下,试验终端可将计算密集型任务卸载至邻近的卫星边缘节点,节点完成计算后再将结果回传,从而实现"数据不做地,信息落下地"的高效处理模式。

随着天基网络承载的试验任务日益复杂,如频谱监测、故障根因分析等,任务之间对计算资源的竞争愈发激烈。传统的调度方法通常假设全网状态已知,要求终端将任务请求发送到中心节点进行统一规划。然而,这种方法在天基环境中会引发一系列问题,特别是在资源竞争与状态感知领域。一方面,卫星节点的资源(计算、存储、能量)远不如地面服务器充沛,多任务并发执行会产生复杂的"并发开销"(Concurrency Overhead),导致在办任务性能急剧下降;另一方面,天基网络拓扑高度动态,维护全局一致的实时状态视图需要消耗大量宝贵的信令开销。

近年来,区块链技术与人工智能(AI)的融合为解决上述问题提供了新的思路。区块链以其去中心化、不可篡改的特性,能够为分布式天基网络提供可信的数据底座与非实时状态共享机制;而联邦学习(Federated Learning, FL)与深度强化学习(Deep Reinforcement Learning, DRL)的结合,则允许在保护数据隐私的前提下,通过分布式协作训练来实现智能化的资源调度与负载预测。这种技术融合方案旨在解决资源受限下的并发竞争与隐私保护之间的矛盾,既充分利用了卫星节点的边缘算力,又有效保障了试验数据的安全与调度决策的科学性。

由于天基网络环境的特殊性,如卫星节点的严格资源受限(SWaP约束)、网络拓扑的高动态变化以及试验数据的多源异构特征,因此,在天基网络场景下应用上述融合技术仍面临一系列亟需解决的挑战:

(一)资源受限下的并发竞争与开销量化:传统的任务卸载研究通常假设边缘节点资源充足,或者仅考虑静态的任务排队模型。然而,在天基边缘计算环境中,卫星节点的计算资源(CPU/GPU)和能量预算极其有限。当多个试验终端同时将任务卸载至同一卫星节点时,新任务会与正在执行的任务争夺计算资源,产生不可忽视的"并发开销",导致所有任务的处理时延增加。不同类型的任务(如图像处理与逻辑计算)对资源的占用特征各不相同,这种异质性使得并发开销难以用简单的数学模型精确量化。如果调度策略未能准确预测并考量这一开销,盲目的任务卸载反而会造成卫星节点过载,降低系统整体吞吐量。

(二)高动态环境下的分布式决策与状态感知:与车联网类似,天基网络具有显著的拓扑动态性。卫星以高速沿轨道运动,与地面终端及邻近卫星的连接关系随时间快速变化。在这种环境下,任何单一节点都难以实时获取全局精确的网络负载状态。传统的基于全局信息的集中式调度算法需要频繁的状态交互,这将消耗大量的星间链路带宽并带来显著的决策滞后。因此,如何利用区块链等技术构建一种轻量级的状态共享机制,并结合深度强化学习(DQN)使终端能够仅凭本地观测和非实时共享信息做出最优的分布式卸载决策,是实现天基网络智能化管控的关键挑战。

(三)数据隐私与可信存证:天基网络试验任务涉及多用户、多部门及海量异构终端,数据来源分散且包含敏感的载荷状态与位置信息。直接上传原始数据进行集中式模型训练存在严重的安全隐患,可能导致敏感试验数据泄露。虽然联邦学习通过仅传输模型参数降低了隐私泄露风险,但在天基网络的开放式信道中,模型参数仍可能面临推理攻击或篡改威胁。此外,在分布式的大规模外场试验中,缺乏统一的可信存证机制会导致任务执行记录易被篡改、故障责任难以界定。因此,亟需构建一种融合区块链与联邦学习的可信架构,在保障数据隐私的同时,实现试验任务全生命周期的可信记录与跨域一致性维护。

因此,在天基网络环境中应用"区块链+联邦学习+强化学习"的融合方法需要克服并发资源竞争、动态分布式决策、数据隐私与可信存证等多个方面的挑战。这些问题相互关联且具有较强的现实意义,直接影响着天基网络试验任务的执行效率与安全性。本文针对上述挑战,展开了面向天基网络分布式试验任务的智能管控与计算卸载策略优化研究。

\section{国内外研究现状及发展趋势}
%\label{sec:***} 可标注label

\subsection{联邦学习研究现状}

McMahan 等人在\cite{McMahan2017}中提出联邦学习(Federated Learning, FL)框架,为"数据不出域"的分布式协同建模提供了基础范式。其核心思想是在各参与节点本地完成模型训练,仅上传模型参数或梯度至聚合端进行全局聚合,从而在避免传输原始数据的同时实现群体智能。将该思想迁移至天基网络/卫星边缘计算场景,联邦学习能够在星载计算资源受限、星地/星间链路带宽紧张且连接窗口间歇的条件下,支持多卫星终端围绕试验任务管控与资源调度所需的模型开展协同训练,例如:面向计算卸载的资源态势预测模型、面向并发资源竞争的"并发开销(Concurrency Overhead)"预测模型等(与本文后续"并发开销建模 + 调度决策优化"的技术路线一致)。

然而,天基网络具备"拓扑时变、资源强约束、数据分布高度异质、通信窗口受限"等典型特性,使联邦学习在该场景落地仍面临多方面挑战:一方面,不同轨道位置、覆盖区域与任务类型产生的数据统计特性差异显著,呈现强非独立同分布(Non-IID);另一方面,卫星节点的算力/能耗预算与链路条件差异巨大,导致训练参与能力与通信代价高度不均;此外,开放式多主体协同下的可信性与安全性问题进一步放大了联邦学习的部署难度。针对上述问题,近年来相关研究主要从"异质性与资源优化""隐私安全与可信聚合"两条主线展开。

\subsection{联邦学习研究的主要方向}

\subsubsection{面向异质性与资源优化的联邦学习研究}

(1)通过算法改造缓解数据统计异质性与多任务共存带来的性能退化。
在天基试验场景中,不同卫星终端常面向不同业务或试验任务(例如:遥感图像预处理、通信信号处理、任务日志统计等),相应训练数据的分布差异显著。若直接采用单一全局模型进行聚合,容易出现全局模型"平均化"导致的收敛变慢与精度下降。针对多分布/多任务环境,Sattler 等人提出聚类联邦学习(Clustered Federated Learning, CFL)\cite{Sattler2020},通过挖掘节点间更新方向或统计特征相似性,将节点划分为若干协作簇并在簇内聚合,从而在非IID场景下提升模型有效性。该类思路对天基网络具有直接借鉴意义:可依据任务类型、资源负载形态或通信窗口模式对卫星节点进行聚类,使模型更贴近"同类节点"的局部统计规律,进而提升诸如并发开销预测、资源态势估计等模型的可用性与泛化能力。

(2)通过参与方选择与资源分配优化提升训练效率并降低通信代价。
在天基网络中,训练轮次受到星地/星间可见窗口、链路带宽、节点算力与能耗预算的共同约束,盲目扩大参与规模往往带来更高通信开销与更差的端到端效率。因此,动态参与者选择成为关键方向。Xiao 等人在\cite{Xiao2021}中围绕边缘环境提出面向训练效率的节点选择与资源优化框架,根据节点能力与链路状态筛选参与方,以加速收敛并降低代价。进一步地,Li 等人在\cite{Li2022}中将强化学习引入联邦学习资源分配与参与控制,使系统能够在动态网络条件下自适应地学习"谁参与、何时参与、分配多少资源"的策略。上述工作对天基试验场景具有启发:在通信窗口短、链路波动大的情况下,需要将"训练参与决策"与"卸载/调度决策"协同考虑,以避免训练过程与业务任务争抢稀缺的计算与通信资源,从而实现整体收益最大化。

\subsubsection{面向隐私安全与可信性的联邦学习研究}

(1)结合区块链构建去中心化可信联邦学习架构。
传统联邦学习依赖中心化聚合服务器,存在单点故障与信任集中风险。在多主体参与、跨域协同的天基试验场景中,聚合端的可信性与审计需求更为突出。Chai 等人在\cite{Chai2021}中提出将区块链引入联邦学习过程,通过链上记录与共识机制实现模型更新的可追溯与不可篡改,从而增强聚合过程的透明性与可信性。该类"区块链 + 联邦学习"的融合范式与天基试验对试验过程记录、模型更新审计、跨域协同一致性的要求天然契合:链上可承载模型版本、更新摘要与关键训练元信息,链下承载模型参数传输与聚合计算,从而在降低信任成本的同时满足工程可落地性。

(2)提升聚合鲁棒性以抵御异常/恶意更新对全局模型的破坏。
在开放分布式环境中,节点可能因策略偏离、数据分布极端或恶意行为上传异常更新,造成全局模型发散或性能劣化。Chen 等人在\cite{Chen2021}中研究了具备拜占庭容错能力的去中心化联邦学习机制,通过鲁棒聚合与一致性校验削弱异常更新的影响。对于天基试验场景而言,该类研究的价值主要体现在:在多主体协同训练中,系统需要具备对"异常更新"带来的训练风险进行抑制的能力,以保障训练稳定性与模型可用性(此处关注的是联邦训练过程的安全稳健性,而非故障检测/诊断任务本身)。

\subsection{分布式任务卸载与智能调度技术研究现状}

随着卫星互联网与车联网(IoV)技术的飞速发展,分布式任务卸载与智能调度已成为解决终端设备"算力赤字"与网络动态变化矛盾的关键手段。在资源受限且拓扑剧烈变动的环境下,国内外研究者围绕任务规划、AI驱动调度及区块链协同等方面开展了广泛探索。

在任务智能规划领域,国外研究起步较早。英国帝国理工学院提出了基于强化学习的自动机器学习框架 AutoML,用于自动选择模型和优化超参数;哈佛大学开发了 Gryffin 算法,实现了高维参数空间下的试验设计优化;瑞士苏黎世联邦理工学院则研制了 LabScheduler 系统,通过约束优化实现了实验室资源的动态调度。国内方面,浙江大学提出了一种基于强化学习的试验智能规划框架,能够自动生成高效试验流程并合理分配计算资源;中国科学院自动化研究所则探索了基于启发式搜索的计划生成方法,提升了试验工作流的生成速度。

针对车联网(IoV)及边缘计算场景,任务卸载决策的复杂性进一步增加。Zhang 等人在\cite{Zhang2021}中提出了一种名为 Moscato 的分布式任务卸载策略,该方案创新性地将客户端车辆转化为区块链共识节点,利用分布式账本实现全球信息的非实时共享,并通过整合联邦学习(FL)与深度 Q 网络(DQN)来平衡新任务与在执行任务之间的资源竞争。Hao 等人在\cite{Hao2021}中针对车联视觉数据的异构性,提出了基于地理感知的联邦学习框架(GeoFed),通过地理聚类机制将车辆划分为协作集群,并利用 DQN 算法动态优化设备选择策略。此外,针对新能源车辆的能效优化问题,Hao 等人进一步设计了能量感知客户端选择算法(e-Fed),通过构建融合算力、能量预算和链路质量的多维约束模型,显著提升了资源利用效率。

在安全性与可信调度方面,区块链技术被广泛用于构建去中心化的信任底座。Du 等人在\cite{Du2021}中研究了基于联盟链增强的移动边缘计算(MEC)系统,提出了一种改进的拜占庭容错协议(IPBFT),并利用近端策略优化(PPO)算法实现了任务卸载决策与资源分配的联合优化。Tang 等人则在\cite{Tang2021}中提出了基于区块链的受信流量卸载框架,通过增强的 EPBFT 共识机制确保了高度动态环境下的卸载安全,并采用联邦强化学习算法提升了系统在存在恶意节点时的鲁棒性。

在面向空天地一体化网络(SAGIN)的动态卸载研究中,研究者们开始关注设备动力学与网络动态的深度耦合。在\cite{SAGIN2021}中提出的 DOGS 机制将卸载问题建模为随机博弈过程,利用多智能体熵增强随机学习算法(MESL),在无需频繁信息交换的前提下降低了 IoT 设备的延迟与能耗。针对复杂物流场景,MT-OSF 框架利用层次分析法(AHP)计算边缘节点优先级,实现了时延敏感任务与计算密集任务的差异化调度,有效降低了任务失败率。

然而,现有的研究工作中,大多数调度策略仍假设网络环境在短时间内处于准静态,且在处理超大规模节点接入时面临信令开销过大、隐私保护与模型效用难以兼顾等挑战。在高度动态的 6G 及算力网络环境下,如何实现实时、泛化且绿色节能的智能调度策略,仍是当前亟待解决的关键问题。

\section{研究内容与主要贡献}

近年来,天基信息网络正由"通信中继网络"向"计算—通信—数据一体化"的边缘计算网络演进。面向天基网络外场试验场景,卫星终端既承担数据产生与任务发起,又受限于星载算力、能量预算与星间/星地链路窗口等条件;在多终端并发任务持续涌入时,边缘节点容易出现资源争用与排队拥塞,从而引发任务时延劣化与服务质量下降。与此同时,试验任务与过程数据跨域分散,多主体协作下难以形成统一可信的记录与审计链条,使得"可信管控"与"高效调度"难以同时满足。针对上述问题,本文面向天基网络分布式试验智能管控与计算卸载需求,构建一套融合区块链、联邦学习与深度强化学习的协同优化框架,实现"可信存证—状态共享—开销感知—智能卸载"的闭环。本文主要研究内容与贡献如下。

(1)提出面向天基试验的分层联盟链可信管控与存证机制。
针对试验任务、过程日志与关键结果分散存储、跨主体协作缺乏统一可信账本的问题,本文设计了"主链—子链"分层多域联盟链架构:主链承载试验任务编排、关键状态摘要与审计索引,子链承载与具体任务相关的存证数据索引与访问策略,实现任务管控信息与业务数据的逻辑解耦。进一步地,本文利用智能合约将任务发布、资源申请、执行回执与结果归档等关键环节固化为链上可验证流程,形成可审计、可追溯的试验证据链。在此基础上,构建一种基于账本广播的非实时全局状态共享机制,为后续分布式卸载决策提供可验证的参考信息,从机制层降低多主体协作中的信任成本与一致性维护开销。

(2)构建联邦学习驱动的并发开销建模方法,为卸载决策提供可计算的"开销先验"。
针对并发任务引发的资源竞争效应难以解析量化的问题,本文将"并发开销(Concurrency Overhead)"作为卸载决策中的关键环境因素,采用学习式建模路线:以边缘节点历史运行数据为基础,训练并发开销预测模型,刻画"节点负载状态—任务画像—时延增量"之间的映射关系。考虑到试验数据的敏感性与链路带宽受限,本文引入联邦学习框架,使各节点在不上传原始数据的前提下完成模型协同训练,从而在满足隐私约束的同时提升模型泛化能力。该模型为后续强化学习卸载策略提供可计算的开销估计与决策反馈,避免仅依赖静态规则或理想化排队假设。

(3)设计面向时变网络与资源竞争的分布式智能卸载策略(FL + DQN 协同)。
在上述并发开销模型基础上,本文进一步构建基于深度 Q 网络(DQN)的分布式卸载决策方法。该方法以任务需求(数据量/计算量/时限)、节点负载态势(队列/可用算力)与链路可用性(窗口/带宽)等为状态输入,在"本地执行—卸载至候选节点—延迟至可用窗口"等动作集合中进行决策,并以任务完成时延与资源竞争代价为主要优化驱动,实现对动态环境的在线自适应。与传统依赖全局实时透明信息的集中式策略相比,该方法通过"账本非实时状态共享 + 学习式开销量化 + DQN 在线决策"的组合,降低了频繁状态同步带来的通信开销,同时提升了并发负载下的决策有效性。

(4)实现并验证天基网络分布式试验智能管控原型系统。
为验证所提机制与算法的可用性,本文实现了一套面向天基试验场景的原型系统,包含分层联盟链存证模块、联邦训练与模型管理模块、DQN 卸载决策模块与可视化监控模块。系统可在典型试验任务负载与多节点协作条件下完成任务发布、链上存证、状态共享与卸载调度的闭环验证。通过系统级实验与对比评估,验证所提方案在任务时延与资源竞争控制方面的改进效果,并展示其在隐私约束与通信受限条件下的工程可行性。

\section{论文结构安排}

本文共分为六章,各章内容安排如下(结构示意图见图 1.1):

\textbf{第一章 绪论。}介绍研究背景与问题来源,分析天基网络试验场景中的可信管控与智能卸载需求;综述区块链存证、联邦学习与强化学习卸载决策等相关研究现状,指出现有方法在多主体可信协同、非实时状态共享与并发资源竞争建模方面的不足;明确本文研究内容、技术路线与主要贡献。

\textbf{第二章 相关理论与技术基础。}介绍天基网络/卫星边缘计算的基本架构与任务卸载模型要素;阐述联盟链与智能合约的关键机制;给出联邦学习聚合流程与典型异质性问题;介绍 DQN 的基本原理与在卸载决策中的建模方式,为后续章节提供理论支撑与符号定义。

\textbf{第三章 面向天基试验的分层联盟链可信存证与状态共享机制。}提出主链—子链分层联盟链架构与跨域协作流程;给出区块结构、关键字段与合约逻辑;设计面向任务管控的链上存证与审计流程,并给出基于账本广播的非实时状态共享机制及其与调度决策的接口方式。

\textbf{第四章 联邦学习驱动的并发开销建模与 FL-DQN 分布式卸载策略。}定义并发开销的建模目标与数据组织方式;构建并发开销预测模型并给出联邦训练流程;在此基础上构建 DQN 卸载决策模型(状态、动作、奖励与训练流程),形成"开销感知—智能决策"的闭环,并通过实验评估其性能。

\textbf{第五章 原型系统实现与系统级验证。}给出原型系统总体架构与模块划分,描述链上存证、联邦训练与卸载决策模块的实现细节与运行流程;构建典型试验任务场景与对比基线,展示系统级评估结果并分析通信开销、隐私约束与并发负载等因素对性能的影响。

\textbf{第六章 结论与展望。}总结全文工作与主要结论,讨论方案的适用条件与局限性,并对面向天基网络试验的可信协同、隐私增强与智能调度等后续研究方向进行展望。
