%%
% BIThesis 研究生学位论文模板 The BIThesis Template for Graduate Thesis
%%

\chapter{基于区块链的试验任务可信管理架构}

\section{本章概述}

本章旨在为融合联邦学习(Federated Learning, FL)、深度强化学习(Deep Reinforcement Learning, DRL)与区块链的天基网络计算卸载策略构建一个系统性、可信且高效的总体架构。在天基网络高动态、资源受限且数据敏感的特殊环境下,单一技术难以应对计算卸载带来的多重挑战。因此,本章重点阐述如何将区块链的去中心化信任机制、联邦学习的隐私保护协同建模能力以及强化学习的动态自适应决策优势有机结合,形成功能完备、逻辑自洽的体系。

首先,在第 3.1 节中,提出一个分层的总体架构,明确数据层、网络层、共识层、合约层及应用层的核心功能与交互接口,并通过系统总体架构图直观展示各组件的协同关系。该设计为后续章节中具体技术和算法的实现提供宏观指导和约束。

其次,在第 3.2 节中,聚焦构建名为“分层联盟链”的可信数据底座。通过设计“主链–子链”结构,实现试验任务全局管理与具体数据存证的有效解耦,在保障数据可信、可追溯的同时,提升系统的吞吐量和可扩展性。本节将详细定义主链和子链的区块结构,并设计一系列关键智能合约以自动化任务的全生命周期管理。

最后,在第 3.3 节中,讨论一种基于区块链账本的非实时状态共享机制。该机制旨在解决分布式决策中的全局信息获取难题,允许卫星终端在不产生高额并发通信开销的前提下,通过查询链上数据获取网络负载、资源占用等“近似”全局状态,并将其作为 DQN 智能体决策的关键输入。这一设计体现了在网络通信资源极端宝贵的天基场景下,对一致性与时效性权衡的工程思路。

通过本章的设计,为后续章节中的并发开销预测模型、分布式卸载决策算法以及系统仿真验证奠定坚实的架构基础。

\section{总体架构设计}

为整合区块链、联邦学习与深度强化学习以应对天基网络计算卸载的挑战,本文设计了一个分层式总体架构,构建集可信存证、隐私保护与智能决策于一体的闭环系统,解决卫星终端在资源受限、高动态变化和数据安全需求下的高效协同问题。

系统总体架构如图 3.1 所示,自下而上分为数据层、网络层、共识层、合约层和应用层。各层职责明确,通过标准化接口协同工作,支撑上层的计算卸载与智能调度应用。

说明:此处应配有系统总体架构图(图 3.1),示意各层组件及其交互关系。

\subsection{各层功能定义}

\paragraph{(1)数据层(Data Layer)}

数据层作为架构基石,负责数据的持久化存储与管理,采用链上链下协同存储策略:
\begin{itemize}
  \item 分层区块数据:核心组成是分层联盟链的区块数据,包括主链和子链。关键且需要强一致性保证的元数据,如任务摘要、状态转换记录、模型更新凭证等,存储在链上。
  \item 链下存储:原始的大容量试验数据及联邦学习模型参数等,通过其哈希值上链的方式,存储于链下分布式存储系统(如 IPFS)或 SES 本地存储中,从而避免区块链膨胀,保证系统性能与可扩展性。
\end{itemize}

\paragraph{(2)网络层(Network Layer)}

网络层负责节点间通信与数据路由:
\begin{itemize}
  \item P2P 网络协议:构建去中心化通信网络,负责节点发现、区块与交易的广播和同步,确保区块链网络的基本连通性。
  \item 跨链网关:实现主链与子链、不同子链之间的信息互通。跨链网关负责处理跨链消息的路由、验证和转发,支撑分层联盟链架构中的数据交互与状态同步。
\end{itemize}

\paragraph{(3)共识层(Consensus Layer)}

共识层确保分布式账本的一致性与不可篡改性,是区块链信任机制的核心:
\begin{itemize}
  \item 共识算法:在许可接入的联盟链环境中,采用投票类拜占庭容错(BFT)算法,如实用拜占庭容错(PBFT)及其变体。这类算法在部分节点失效或作恶的情况下仍能保证全网数据一致性,并且相较工作量证明(PoW)具有更低延迟与更高吞吐量,更适合具有实时性需求的天基网络场景。
\end{itemize}

\paragraph{(4)合约层(Contract Layer)}

合约层定义并执行业务逻辑,实现系统的自动化与智能化管理:
\begin{itemize}
  \item 智能合约:部署在区块链上的可执行代码,用于定义任务管理、资源调度与联邦学习等过程的规则。本文设计四类核心智能合约:
  \begin{itemize}
    \item 任务发布合约:处理任务创建、发布、参数定义与资源需求的链上注册;
    \item 状态更新合约:追踪并记录任务从“已发布”“执行中”到“已完成”等状态的流转;
    \item 结果验证合约:定义任务结果验证规则并记录验证结果,为后续激励与惩罚提供依据;
    \item FL 模型管理合约:管理联邦学习过程,包括模型版本记录、参与方列表维护、全局模型更新摘要存证等,增强联邦学习过程的透明度与可审计性。
  \end{itemize}
\end{itemize}

\paragraph{(5)应用层(Application Layer)}

应用层直接面向用户,是整个架构价值的最终体现:
\begin{itemize}
  \item 天基试验任务应用:用户通过该接口提交试验任务、查询任务状态并获取结果;
  \item 分布式智能卸载决策(DQN):每个卫星终端内嵌 DQN 智能体,通过查询账本获取非实时状态共享信息,结合本地观测做出任务本地执行或卸载至某 SES 的决策;
  \item 联邦学习训练(FL):多个卫星终端之间开展协同建模,各终端使用本地数据训练并发开销预测模型等,仅上传模型参数至聚合节点,在保护隐私前提下构建高精度全局模型。
\end{itemize}

\subsection{数据流与控制流}

系统核心工作流程围绕数据流与控制流展开,可概括为以下三个闭环。

\paragraph{(1)卸载请求与决策流程(控制流)}

\begin{enumerate}
  \item 任务产生:卫星终端在应用层产生计算密集型任务;
  \item 状态查询:终端中的 DQN 智能体通过合约层接口查询数据层账本,获取其他节点任务负载、资源占用等“近似”全局状态;
  \item 智能决策:DQN 智能体结合链上状态与本地观测信息,做出任务本地执行或卸载至某边缘节点的决策;
  \item 任务发布:若决定卸载,终端调用任务发布合约将任务元数据及目标节点信息写入区块链,并通过网络层将原始任务数据发送至目标 SES。
\end{enumerate}

\paragraph{(2)联邦学习与模型更新流程(数据流与控制流)}

\begin{enumerate}
  \item 本地训练:各终端在应用层利用本地采集的任务执行数据(时延、资源占用率等)训练并发开销预测模型;
  \item 模型上传:训练完成后,终端将模型参数更新上传至聚合节点(可为特定卫星或地面站);
  \item 全局聚合:聚合节点执行 FedAvg 等聚合算法生成新的全局模型;
  \item 模型存证与分发:聚合节点调用 FL 模型管理合约,将新全局模型的版本号与哈希摘要记录在链上,随后通过网络层将模型分发给各终端,用于后续卸载决策。
\end{enumerate}

\paragraph{(3)任务执行与账本记录流程(数据流)}

\begin{enumerate}
  \item 任务执行:边缘节点接收任务后开始执行计算;
  \item 状态更新:在任务执行关键阶段(如开始执行、执行完成),边缘节点或任务发起方调用状态更新合约,将最新任务状态写入区块链;
  \item 结果验证与存证:任务完成后,结果由指定验证节点(或多方)通过结果验证合约进行验证。验证通过后,将结果哈希及验证记录存证至主链,原始结果数据存放于链下。
\end{enumerate}

通过上述设计,总体架构将区块链的可信记录能力、联邦学习的隐私保护建模能力与深度强化学习的自适应决策能力深度融合,为天基网络环境下的计算卸载提供了一个完整、高效且可信的解决方案。

\section{分层联盟链可信数据底座}

为高效、可信地管理天基网络中分布式试验任务与海量数据,本文设计了一种基于分层联盟链的可信数据底座。传统单链架构在处理大规模交易和异构数据时易遭遇性能瓶颈和治理复杂性问题,而分层联盟链通过将不同职责的业务逻辑划分到不同链上,实现功能解耦与性能优化,是应对天基网络复杂性的有效方案。

本节设计的核心是“主链–子链”架构,其结构如图 3.2 所示。

说明:此处应配有分层联盟链结构图(图 3.2),展示主链、子链与跨链网关的关系。

\subsection{主链–子链职责边界}

\paragraph{(1)主链(Main Chain)}

主链作为系统的“信任根”和“控制中枢”,由地面控制中心、区域网关等核心节点构成,具有较高安全级别与全局管理权限,其核心职责包括:
\begin{itemize}
  \item 全局任务管理与治理:负责试验任务的发布、分配及最终状态记录,所有任务生命周期关键节点(创建、完成、失败等)在主链上进行最终确认与共识,同时承担节点准入、权限变更等联盟治理职能;
  \item 全局状态摘要与锚定:定期锚定各子链的状态摘要(如最新区块头哈希),实现对子链数据的全局可验证性,而无需在主链上存储全部子链数据,从而显著减轻主链存算负载;
  \item 跨链协调与控制:通过跨链网关管理子链生命周期,包括子链创建、成员准入/退出授权以及跨链交互规则与协议的制定。
\end{itemize}

\paragraph{(2)子链(Sub-chain)}

子链作为“数据存证与执行层”,可按地理区域、业务部门或任务类型(如遥感图像处理、通信信号监测)创建多条,由相关卫星边缘节点与地面站共同维护,其职责包括:
\begin{itemize}
  \item 具体任务数据存证:存储与特定任务相关的操作日志、中间结果、传感器读数等海量高频数据,在子链内直接记录共识,实现业务级快速存证与审计;
  \item 局部共识与执行:在子链参与节点间达成局部共识,局部化处理显著降低交易确认延迟,提升整体吞吐量。
\end{itemize}

\paragraph{(3)跨链网关(Cross-Chain Gateway)}

跨链网关作为主链与子链之间的可信桥梁,通常部署在同时参与主链与相应子链的节点(如区域网关)上,负责:
\begin{itemize}
  \item 监听源链上的特定事件(如主链任务发布事件);
  \item 验证事件的合法性与完整性;
  \item 在目标链上触发相应交易(如在子链创建任务记录),实现安全可靠的链间数据交换与业务协同。
\end{itemize}

这种分层架构通过将全局治理与具体数据处理分离,有效克服单链架构性能瓶颈,在确保全局一致性与数据可追溯性的同时,提供良好可扩展性与工程可落地性。

\subsection{区块结构定义与智能合约流程}

为支撑分层架构的高效运行,本文对主链与子链的区块结构和智能合约流程进行针对性设计,实现任务全生命周期的自动化、透明化与可信化。

\paragraph{(1)区块结构定义}

主链区块主要承载全局任务信息与跨链证明,而子链区块则专注于存储具体试验数据与执行日志。示意性字段定义如下。

\begin{itemize}
  \item 主链区块结构:
  \begin{itemize}
    \item 区块头:包含版本号、时间戳、前一区块哈希、Merkle 树根(聚合任务列表与子链状态树)、Nonce 等标准字段;
    \item 任务列表:记录本区块内发生状态变更的全局任务。每条记录包含任务 ID、发起者标识、执行节点标识、当前状态(发布/执行中/完成/失败)、任务内容摘要等;
    \item 子链状态树根哈希:通过 Merkle 树等形式聚合在该时间窗口内被锚定到主链的各子链状态哈希,用于实现对子链数据的全局可验证性。
  \end{itemize}
  \item 子链区块结构:
  \begin{itemize}
    \item 区块头:包含版本号、时间戳、前一区块哈希、Merkle 树根(交易/数据列表)、指向主链对应任务区块的哈希链接等;
    \item 交易/数据列表:记录详细试验数据或操作日志,每条记录包含试验 ID、操作者、时间戳、数据内容(或链下存储地址哈希)等。
  \end{itemize}
\end{itemize}

\paragraph{(2)基于智能合约的任务生命周期流程}

任务生命周期主要通过主链上的一系列智能合约驱动与管理,确保工程可落地性与性能权衡。其流程如图 3.3 所示。

说明:此处应配有基于智能合约的任务生命周期流程图(图 3.3)。

\begin{enumerate}
  \item 任务发布:试验发起方调用任务发布合约在主链创建新任务。合约验证发起方权限与任务参数合法性后,记录任务 ID、需求、预算等信息,并触发“任务已发布”事件,任务进入待分配状态。
  \item 节点选择与任务执行:卫星边缘节点监听“任务已发布”事件发现新任务,若决定承接则调用合约锁定任务。状态更新合约将任务状态置为“执行中”,并记录承接节点信息;随后边缘节点在其所属子链上记录详细执行日志。
  \item 结果提交与验证:任务完成后,边缘节点将结果哈希及指向子链存证数据的链接提交至主链结果验证合约,合约按预设规则(指定权威节点验证或多方投票等)触发验证流程。
  \item 状态终结与激励:
  \begin{itemize}
    \item 若验证通过,状态更新合约将任务状态标记为“已完成”,并自动触发激励机制(如信誉积分增加、代币支付等);
    \item 若验证失败,任务状态标记为“已失败”,并可触发相应惩罚或仲裁机制。
  \end{itemize}
\end{enumerate}

所有状态最终变更均固化在主链上,构成不可篡改的“最终事实”。通过这套基于智能合约的自动化流程,分层联盟链不仅为天基网络试验提供可信数据存证能力,也大幅提升任务管理效率与透明度,为构建无需中心化信任的分布式协同环境奠定坚实基础。

\section{基于账本的非实时状态共享机制}

在天基网络这种高度动态且通信资源受限的环境中,获取精确、实时的全局网络状态(如所有节点的 CPU 占用率、任务队列长度等)几乎不可能。传统集中式调度方案依赖周期性状态上报,带来巨大信令开销与明显决策延迟,尤其在大规模节点网络中问题突出。为此,本文提出一种基于账本的非实时状态共享机制。

该机制核心思想是:利用区块链账本作为分布式、最终一致的全局状态信息源,允许各决策节点(卫星终端)在无需高频实时通信的情况下,获取“近似”的全局状态视图,并将其作为本地 DQN 智能决策的重要输入。

\subsection{机制原理}

如图 3.4 所示,区块链账本由网络中所有共识节点共同维护。当网络中发生关键事件(如新任务卸载至某边缘节点、任务完成释放资源)时,相应智能合约被触发,将这些状态变更以交易形式写入新区块。随着区块的广播与确认,这些信息被同步到所有节点的本地账本副本中。

说明:此处应配有“节点通过账本感知全局状态”示意图(图 3.4)。

某卫星终端节点(如节点 A)在需要做出卸载决策时,无需向全网广播查询请求,而是仅需:
\begin{enumerate}
  \item 本地查询:直接查询本地维护的账本副本;
  \item 状态重构:解析账本中与任务和资源相关的历史交易记录,重构网络中其他节点在“不久之前”的状态快照。例如,通过扫描最近若干区块的任务发布与完成记录,估算节点 B、C 当前任务队列长度与资源占用情况;
  \item 决策输入:将重构出的“近似”全局状态信息与本地状态(如本地 CPU 负载、电池余量)一同作为 DQN 的状态输入。
\end{enumerate}

\subsection{一致性与时效性的折中原则}

该机制获取的状态并非严格实时,其时效性受出块速度与网络同步延迟影响,本质上体现了在一致性与时效性之间的折中。

\paragraph{(1)强一致性与弱时效性}

区块链保证所有节点最终可见完全一致的状态历史(强最终一致性)。但从事件发生到被打包进区块并完成确认存在固有延迟(在联盟链中一般为秒级),因此节点查询到的状态是“过去”的快照。

\paragraph{(2)查询频率与折中原则}

\begin{itemize}
  \item 原则:查询频率应根据任务时延敏感度与网络动态性自适应调整;
  \item 高时效性需求:对于极度时延敏感任务,决策前可触发一次对最新区块的强制同步,以获取尽可能新的状态,但会增加单次决策的成本;
  \item 一般性需求:对于大多数任务,可在 DQN 固定决策时间步内(如每秒)查询本地账本最新状态,通信开销最低;
  \item 成本–效益:该机制将状态同步成本平摊至区块链网络的日常维护,避免为每次决策单独发起昂贵的实时轮询,在天基网络中尤为重要。
\end{itemize}

\subsection{对 DQN 决策的支撑}

在这种非实时共享机制下,DQN 智能体可以逐步学会在信息不完全且存在延迟的环境中做出鲁棒决策:

\begin{itemize}
  \item 学习含噪环境下策略:DQN 通过大量交互经验学习在带噪声和延迟的状态输入下获得长期最优回报的策略,而非依赖完全精确的瞬时状态;
  \item 时效性加权:在状态表示设计中,可对来自账本的信息进行时效性加权处理,将最新区块信息赋予更高权重,从而引导模型更多关注近期网络变化趋势。
\end{itemize}

综上,基于账本的非实时状态共享机制是针对天基网络特点量身定制的轻量级、高容错全局状态感知方案。它充分利用区块链的数据共享特性,以可接受的时效性损失换取通信开销的大幅降低和系统鲁棒性的提升,为实现真正分布式的智能卸载决策提供了可行的数据基础。

\section{本章小结}

本章围绕天基网络中卫星终端试验任务的可信管理与智能卸载需求,构建了一个融合区块链、联邦学习与深度强化学习的系统级架构。

首先,从层次化视角提出总体架构设计,明确了数据层、网络层、共识层、合约层与应用层的职责划分与协同方式,解释了任务卸载、联邦学习训练与任务执行三大关键流程的数据流与控制流路径。其次,围绕分层联盟链可信数据底座,详细定义主链–子链职责边界及区块结构,并设计基于智能合约的任务生命周期管理流程,实现了试验任务从发布、执行到结果存证的全流程自动化与可审计性。再次,通过引入基于账本的非实时状态共享机制,在保证系统全局一致性与可信性的前提下,大幅降低了获取全局状态的通信开销,为在信息不完全环境下运行的 DQN 智能体提供了可靠的环境感知基础。

本章构建的“分层联盟链可信数据底座 + 非实时状态共享 + 智能合约驱动任务管理”框架,与前一章中的 FL 并发开销预测与 DRL 卸载决策模型形成有机衔接,为后续章节中算法实现与系统仿真验证提供了统一的架构支撑。

% 参考文献可统一在全文末尾给出,这里不单列环境。

