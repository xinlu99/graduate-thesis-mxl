%%
% BIThesis 研究生学位论文模板 The BIThesis Template for Graduate Thesis
% This file has no copyright assigned and is placed in the Public Domain.
%%

\chapter{相关理论与关键技术}

\section{天基网络与卫星终端试验任务的系统架构与挑战}

为了有效应对天基网络中计算卸载的复杂性,首先需要构建一个清晰、准确的系统模型,用以描述网络中的关键实体、它们的交互关系以及所面临的独特环境与挑战。本节将详细定义系统架构,并对其中的关键挑战进行形式化分析。

\subsection{系统架构}

本文所研究的天基网络计算卸载系统架构如图~2.1 所示,是一个典型的三层异构网络,由天基网络层、空口通信层与地面网络层构成,研究的核心则聚焦于天基网络内部的分布式协同。

说明:此处应配有一张系统架构图(例如,包含卫星终端、LEO 边缘卫星、GEO 骨干卫星和地面站的示意图),以直观展示多层结构和实体间的关系。

\paragraph{(1)卫星终端层(Satellite Terminal Layer)}

该层由大量执行各类试验任务的卫星终端(Satellite Terminal, ST)构成。这些终端可以是搭载特定传感器或载荷的科学实验卫星、对地观测卫星,或其他需要在轨进行数据处理的航天器。在本模型中,ST 是计算任务的发起者。每个 ST 具备一定的本地计算能力,但受制于 SWaP-C(Size,Weight,Power,Cost)约束,其计算、存储和能量资源通常有限。当面临计算密集型或时延敏感型任务时,ST 会寻求将任务卸载至更高层级的计算节点。

\paragraph{(2)天基边缘计算层(Space-based Edge Computing Layer)}

该层主要由一组低地球轨道(LEO)卫星构成,它们在承担通信中继功能的同时,配备了增强的计算与存储单元,作为天基边缘服务器(Satellite Edge Server, SES)。这些 SES 节点在轨道上形成一个动态的分布式计算网络,能够为邻近 ST 提供低时延的计算卸载服务,是计算任务的主要执行者。由于 LEO 卫星高速运动,SES 网络的拓扑结构高度动态,星间链路(Inter-Satellite Links, ISL)会频繁建立与断开。

\paragraph{(3)天基核心网络/骨干层(Space-based Core Network / Backbone Layer)}

该层可由地球静止轨道(GEO)卫星或中地球轨道(MEO)卫星构成,它们拥有更强的处理能力和更广的覆盖范围,作为区域性的数据汇聚与交换中心,或充当与地面网络的网关。在某些场景下,这一层的节点也可以作为计算卸载的备选执行节点。

在本研究中,我们主要关注卫星终端层与天基边缘计算层之间的交互关系。一个 ST 可以将任务卸载至其通信范围内任意一个 SES,系统的核心目标是构建一个去中心化的资源管理与任务调度机制,使得每个 ST 能够智能地决策:是将任务在本地执行,还是将其卸载至哪个最合适的 SES 上执行。

\subsection{关键挑战的形式化描述}

基于上述系统架构,天基网络计算卸载面临的核心挑战可形式化描述如下。

\paragraph{(1)网络拓扑的高动态性(High Dynamics of Network Topology)}

LEO 卫星以约 \(7.8~\text{km/s}\) 的高速度绕地运行,导致 ST 与 SES 之间以及 SES 与 SES 之间的可见时间窗口短暂且不断变化,从而使任意节点 \(i\) 在时刻 \(t\) 的邻居集合
\[
N_i(t)
\]
呈显著的时变性。这种高动态性对计算卸载策略的连续性和稳定性提出了严峻考验。例如,一个正在向某个 SES 卸载数据的 ST,可能因该 SES 飞出通信范围而导致卸载中断。已有研究如 Zhao 等人文献~[12] 中提出的时空注意力 DRL 方法,正是为了刻画与利用这种时空动态特性。

\paragraph{(2)资源状态信息的不完全性(Incomplete Information of Resource Status)}

在高动态、大时延的天基网络中,任何单个节点都难以实时、精确地获取全局网络状态。对于一个需要做出卸载决策的 ST,它能够观测到的其他 SES 节点的计算负载、任务队列长度等信息,本质上是延迟的、不完整的,即存在明显的部分可观测性(Partial Observability)问题。此时,传统基于“全局信息透明”假设的集中式优化算法难以奏效,而在信息不对称条件下进行决策的分布式方法(如文献~[17] 的相关工作)更贴近真实场景,这也驱动我们必须寻求仅依赖局部或异步共享信息的分布式决策范式。

\paragraph{(3)任务并发执行的性能影响(Performance Impact of Concurrent Task Execution)}

当多个 ST 同时向同一 SES 卸载任务时,这些任务将在 SES 上并发执行,共同竞争 CPU、内存、I/O 等计算资源。与理想化的线性资源划分模型不同,实际并发执行会引入额外性能开销,如缓存争用(Cache Contention)、上下文切换(Context Switching)等,导致每个任务的实际完成时间均高于其独占运行的情形。这种现象被称为“并发开销”(Concurrency Overhead)[14]。若忽略并发开销,将严重低估任务完成时延,进而做出次优甚至错误的卸载决策。准确建模与预测并发开销,是实现精细化资源管理与高效卸载的关键前提。

\paragraph{(4)分布式协同的信任缺失(Lack of Trust in Distributed Collaboration)}

天基网络天然是由不同所有者、不同管理域的节点构成的分布式系统。在缺乏中心化可信第三方的情况下,如何确保任务数据的完整性、计算结果的正确性以及资源计费的公平性,是一个棘手的问题。例如,一个恶意 SES 可能返回伪造的计算结果,而恶意 ST 可能谎报任务资源需求。因而需要一种技术机制来建立节点间的信任,保障数据与计算过程的可追溯、不可篡改、可审计,这也是文献~[2, 4, 5] 中区块链应用被广泛讨论的核心动机之一。

综上,天基网络计算卸载的核心科学问题可归结为:在资源受限、拓扑高动态、信息不完全且信任缺失的分布式环境下,如何设计智能、高效且可信的计算卸载与资源调度策略,以实现系统整体效能(如时延、能耗、吞吐量等)的最优化。

\section{基于区块链的分布式可信数据管理与协作}

为应对前述分布式协同中的信任缺失挑战,本文引入区块链技术,构建一个贯穿任务全生命周期的可信数据管理与协作框架。区块链的去中心化、不可篡改、公开透明和可追溯特性,使其成为解决分布式系统信任问题的理想技术方案[2]。本节将详细阐述面向天基网络环境定制的区块链系统设计,包括其分层架构、核心机制及工作流程。

\subsection{分层联盟链架构}

考虑到天基网络资源受限的特点,尤其是星载计算与存储资源的珍贵,传统公有链(如比特币)采用的工作量证明(Proof of Work, PoW)共识机制因其巨大的算力与能耗开销而完全不适用。同时,若将所有试验任务的原始数据或过程数据全部上链,将迅速耗尽星上存储并造成巨大的通信负担。

为此,本文设计了一种面向天基试验任务的分层联盟链(Hierarchical Consortium Blockchain)架构,并结合链上链下(On-chain / Off-chain)协同存储机制。该架构的设计思想借鉴了文献~[5] 的分层思路以及作者前期在车联网领域的研究[20],旨在实现控制信令与海量数据的分离,在确保关键信息可信、可追溯的同时,最大化系统运行效率与可扩展性。

\paragraph{(1)联盟链(Consortium Blockchain)}

本文选用联盟链作为底层技术形态。在联盟链中,节点的加入需要预先授权,只有获得许可的卫星终端(ST)、天基边缘服务器(SES)和地面站(Ground Station, GS)才能成为链的参与者。这既保证了系统的安全可控,又便于采用轻量级共识机制,如权威证明(Proof of Authority, PoA)、实用拜占庭容错(PBFT)等,其计算开销远低于 PoW,更适用于资源受限的卫星节点[4]。

\paragraph{(2)分层架构(Hierarchical Architecture)}

如图~2.2 所示,联盟链架构分为主链与子链两个层次:

\begin{itemize}
  \item 主链(Main Chain):主链是系统的“任务控制链”或“全局状态总账”,其核心功能是记录试验任务相关的关键元数据与控制信令。一笔典型“任务交易”可能包含:任务 ID、发起方 ST 的公钥、任务类型、计算量需求、数据大小、目标 SES 的公钥、任务发起时间戳、任务完成时间戳以及链下数据的存储哈希指针等。主链由系统中所有或部分核心节点(如 SES 和 GS)共同维护,负责全局任务协调、状态同步与最终结果的共识确认。
  \item 子链(Sub-chain)/ 状态通道(State Channel):子链是可选的、面向具体任务或区域的“数据存证链”,可由参与特定复杂任务的多个节点临时组建,用于高频记录中间过程数据。更轻量级的方式是采用状态通道技术,在链下进行大量快速状态更新,仅将初始状态与最终状态上链。这种设计将大量非关键过程数据由链下或子链处理,显著减轻主链负担。
\end{itemize}

说明:此处应配有一张分层联盟链架构图(图~2.2),展示主链和子链的关系,以及 ST、SES、GS 等节点在链上的角色和数据流向。

\paragraph{(3)链上链下协同存储}

对于试验任务产生的海量原始数据和结果数据,本文采用链上链下协同方式进行存储。链上仅存储数据的哈希值(Hash)及描述其元数据的索引信息,而数据本体则存储在 SES 本地存储或专用分布式存储系统(如 IPFS)中。通过将数据哈希锚定在不可篡改的区块链上,一方面可保障数据完整性与可追溯性(任意链下数据篡改都会导致哈希不匹配),另一方面避免链上存储膨胀,实现效率与安全的平衡。

\subsection{核心机制:智能合约与非实时状态共享}

\paragraph{(1)基于智能合约的自动化任务管理}

智能合约(Smart Contract)是部署在区块链上、可按预设规则自动执行的程序。在本架构中,智能合约用于实现试验任务全生命周期的自动化与去信任化管理[6]。

\begin{itemize}
  \item 任务生命周期合约:可编写智能合约定义试验任务从发布、竞标(可选)、分配、执行到结果提交与验证的完整流程。例如,当 ST 发布任务时,触发合约函数将任务信息记录上链;当 SES 完成计算并提交结果哈希时,触发另一函数,由指定验证节点(或多方投票)对结果进行校验。校验通过后,合约可自动完成资源计费与信誉评分更新。
  \item 访问控制合约:通过智能合约可实现对试验数据与计算资源的精细、动态访问控制。合约可规定只有满足特定条件(如具备特定身份或位于特定区域)的节点才可访问相应数据或调用特定计算服务。
\end{itemize}

\paragraph{(2)基于区块链的非实时全局状态共享}

区块链作为追加式(Append-only)的分布式账本,天然为所有参与节点提供最终一致的全局状态视图。一个 ST 在决定是否卸载新任务时,无需向全网广播查询各 SES 的实时负载,而只需读取主链最新区块,即可获知:

\begin{itemize}
  \item 当前有哪些任务正在哪些 SES 上执行(通过比对尚未匹配“任务结束”记录的“任务开始”记录);
  \item 各 SES 的历史负载情况与平均处理时长;
  \item 不同节点的信誉评级。
\end{itemize}

尽管这些信息并非严格实时(新区块产生存在时间间隔),但对于计算卸载这类“准实时”决策而言,这种非实时全局状态已足以提供决策所需的宏观环境感知。它极大降低了获取系统状态的通信开销,并缓解了网络时延导致的信息不一致问题,是实现分布式智能决策的关键数据基础。这一思想与文献~[17] 中针对信息不对称问题的研究是不谋而合的,并通过区块链提供了可工程落地的实现路径。

通过上述设计,本文构建了一个面向天基网络优化的分布式可信数据管理框架:利用轻量级联盟链保障系统安全可控与高效运行,通过分层架构与链上链下协同解决海量数据存储难题,并借助智能合约与非实时状态共享机制,为上层联邦学习与深度强化学习应用提供可靠、透明、高效的协同底座。

\section{联邦学习的并发开销预测与隐私协作建模}

在获得可信、非实时的全局状态视图之后,卸载决策面临的下一个核心难题是:如何准确评估一个卸载动作的“成本”,即任务的预期完成时延。如第~\ref{chap:intro} 章中所述,由于“并发开销”的存在,任务时延并非简单等于其计算量除以节点剩余计算能力,而是一个受当前节点负载、任务类型组合等多因素影响的复杂非线性函数。本节阐述如何利用联邦学习(Federated Learning, FL)技术,以隐私保护方式协作构建高精度的并发开销预测模型。

\subsection{并发开销的数据驱动建模}

并发开销产生于现代计算机体系结构中的共享资源竞争,包括 CPU 多级缓存、内存总线、磁盘 I/O 等。当多个计算密集型任务同时运行时,它们会频繁驱逐彼此的缓存行、争抢内存带宽,导致处理器流水线停顿与频繁缺页中断,使每个任务的“每秒有效指令数”显著低于独占 CPU 时的水平。

这种性能下降程度受到多种因素影响,例如:

\begin{itemize}
  \item 硬件架构:不同型号 CPU 的缓存大小、核心数量、内存控制器性能各不相同;
  \item 任务类型:访存密集型任务与计算密集型任务混合执行时,与同类任务混合执行相比,其干扰模式显著不同;
  \item 任务数量与组合:并发任务数量越多,资源竞争通常越激烈;
  \item 操作系统调度策略:调度器如何分配时间片与 CPU 核心,直接影响任务间相互影响。
\end{itemize}

由于上述因素高度交织,试图仅依靠解析方法(如排队论)精确建模并发开销极为困难,文献~[14] 已对此有深入讨论。因此,更可行的方法是采用数据驱动思路:将并发开销量化为待学习函数 \(f_{\text{co}}\),通过大量历史运行数据拟合:
\[
(L_{\text{new}}, \Delta L_{\text{exist}}) = f_{\text{co}}(\text{NodeSpec}, \text{TaskSet}_{\text{current}}, \text{Task}_{\text{new}}),
\]
其中,\(L_{\text{new}}\) 为新任务预计完成时延,\(\Delta L_{\text{exist}}\) 为已在执行任务的预期额外时延,\(\text{NodeSpec}\) 为节点硬件规格,\(\text{TaskSet}_{\text{current}}\) 为当前正在执行的任务集合,\(\text{Task}_{\text{new}}\) 为新卸载任务。

在天基网络中,每个 SES 节点本地可用的历史运行数据有限,形成典型的“数据孤岛”。仅凭本地数据训练的模型,泛化能力与预测精度难以保证;而若将所有 SES 数据集中到服务器训练,又会面临巨大的通信开销与严重隐私泄露风险。

\subsection{基于联邦学习的协作式预测模型训练}

联邦学习作为“数据不动、模型动”的分布式机器学习范式,为解决上述矛盾提供了理想框架[3]。其核心思想是:参与方(此处为各 SES)在不共享本地原始数据的前提下,通过交换与聚合本地训练模型参数,共同训练出全局模型,这与 Zhai 等人为 LEO 卫星网络设计的去中心化 FL 框架 FedLEO[7] 的动机高度一致。

本文提出的基于 FL 的并发开销预测模型训练流程如下(如图~2.3 所示):

说明:此处应配有一张联邦学习流程图(图~2.3),展示中心协调器(如地面站或指定 GEO 卫星)与多个 SES 节点之间的模型分发、本地训练与模型上传聚合过程。

\begin{enumerate}
  \item \textbf{初始化:}由中心协调器(可为地面站或骨干卫星)初始化并发开销预测模型(如 CNN 或 MLP),并将参数分发给所有参与 SES。
  \item \textbf{本地训练:}每个 SES 节点 \(k\) 使用本地历史任务运行日志 \(D_k\) 作为训练集。每条日志记录包含任务组合、节点负载及最终实际时延等信息。节点 \(k\) 在此基础上利用随机梯度下降(SGD)等算法对本地模型 \(w_k\) 进行多轮训练,以最小化预测时延与真实时延的误差。
  \item \textbf{模型上传:}本地训练结束后,每个 SES 节点 \(k\) 将更新后的模型参数(或增量)\(w_k\) 发送给协调器。传输过程中只包含模型参数,不包含任何原始运行数据,从而保护本地数据隐私。
  \item \textbf{全局聚合:}协调器收集多个(或全部) SES 的模型参数后,采用聚合算法生成新一轮全局模型。最常用的是联邦平均(FedAvg)算法,其更新公式为
  \[
    w_{\text{global}}^{t+1} = \sum_{k=1}^{K} \frac{n_k}{N} w_k^{t+1},
  \]
  其中 \(K\) 为参与本轮聚合节点数,\(n_k\) 为节点 \(k\) 的本地样本量,\(N\) 为所有参与节点的总样本量。
  \item \textbf{模型分发:}协调器将聚合后的全局模型参数 \(w_{\text{global}}^{t+1}\) 再次分发给所有 SES,替换其本地模型。
\end{enumerate}

通过多轮上述迭代,系统可收敛得到融合各节点数据特征的高精度全局并发开销预测模型。

\subsection{关键技术考量}

在将 FL 应用于天基网络实践时,还需重点考虑以下问题:

\begin{itemize}
  \item \textbf{客户端选择(Client Selection):}每轮训练中不必要求所有 SES 参与。尤其在卫星场景中,部分节点的星地/星间链路条件较差甚至不可用,需要设计合理的参与方选择策略,优先选择信道条件较好、算力充足且数据质量高的节点参与训练,以平衡模型性能和通信开销。
  \item \textbf{异构性问题(Heterogeneity):}天基网络节点存在系统异构(硬件配置、算力差异)与数据异构(任务类型与负载模式不同,Non-IID 数据)。这可能导致本地模型收敛速度与方向不一致,影响全局模型性能。可采用 FedProx 等算法,在本地损失函数中加入正则项限制本地更新偏离全局模型过远,或采用分层/聚类联邦学习[8] 等方法缓解异构性影响。
  \item \textbf{增强隐私保护:}尽管 FL 不传输原始数据,但模型更新仍可能泄露部分隐私。可结合差分隐私(Differential Privacy)技术,在 SES 上传模型参数前加噪,实现可量化的更强隐私保障[9]。
\end{itemize}

通过引入联邦学习,本文在保护各节点数据隐私的前提下,提出了一种针对“并发开销”精准预测的协作建模方案。得到的全局预测模型将作为下一节深度强化学习智能体的环境感知模块,为其提供更准确的成本预估,从而支撑更优的动态卸载决策。

\section{深度强化学习的动态资源调度与卸载节点优选}

面对天基网络拓扑高动态、资源状态不确定等核心挑战,传统基于静态优化或启发式规则的卸载算法难以适应。深度强化学习(Deep Reinforcement Learning, DRL)作为一种无模型(Model-free)的决策优化方法,通过让智能体在与环境的持续交互中“试错”学习,能够自主发现复杂动态约束下的近似最优策略,是解决此类问题的理想工具[18, 10]。本节将计算卸载问题形式化为马尔可夫决策过程(MDP),并阐明如何利用 DRL 进行求解。

\subsection{计算卸载的马尔可夫决策过程建模}

我们将单个卫星终端 ST 的计算卸载决策过程建模为一个 MDP,其核心要素定义如下。

\paragraph{(1)智能体(Agent)}

每个需要进行卸载决策的 ST 视为一个独立智能体。

\paragraph{(2)状态(State, \(S\))}

状态是智能体在决策时可观测的环境信息,合理的状态空间设计是 DRL 成功的关键。时刻 \(t\) 的状态记为 \(s_t\),包括:
\begin{itemize}
  \item 任务属性:\(s_t^{\text{task}} = \{C_i, D_i\}\),其中 \(C_i\) 为当前任务 \(i\) 的计算量(如 CPU 周期数),\(D_i\) 为其数据大小;
  \item 本地节点状态:\(s_t^{\text{local}} = \{E_{\text{local}}\}\),表示本地节点剩余能量;
  \item 网络环境状态:\(s_t^{\text{net}} = \{Q_j, B_j, L_j\}_{j \in N(t)}\),其中 \(N(t)\) 为当前可达邻近 SES 集合,\(Q_j\) 为通过区块链获取的 SES\(_j\) 任务队列信息(如队长、任务数),\(B_j\) 为通过广播或链路探测获得的信道状态(带宽、信噪比等),\(L_j\) 为利用第~2.3 节基于 FL 训练的并发开销预测模型得到的在 SES\(_j\) 上执行任务的预计时延。
\end{itemize}

综合而言,
\[
s_t = \{s_t^{\text{task}}, s_t^{\text{local}}, s_t^{\text{net}}\}.
\]
这一状态设计通过区块链与联邦学习融合了更丰富、准确的环境信息,有效缓解了部分可观测性问题。

\paragraph{(3)动作(Action, \(A\))}

动作是智能体在给定状态下可采取的决策。对于一个拥有 \(M\) 个可达 SES 节点的 ST,其动作空间为离散集合:
\[
A = \{a_0, a_1, \dots, a_M\},
\]
其中,\(a_0\) 表示本地执行任务,而 \(a_j\)(\(j=1,\dots,M\))表示将任务卸载至第 \(j\) 个 SES 节点。这一建模方式已在多篇任务卸载研究中得到采用[11, 13]。

\paragraph{(4)奖励(Reward, \(R\))}

奖励函数用于评估在某状态下采取某动作的即时优劣,直接引导智能体学习方向。为实现系统性能综合优化,本文设计多目标奖励函数。在状态 \(s_t\) 下采取动作 \(a_t\),系统转移至新状态 \(s_{t+1}\) 后,智能体获得奖励:
\[
r_t = - \left(\omega_T T_{\text{total}} + \omega_E E_{\text{cost}} + \omega_B B_{\text{imbalance}}\right),
\]
其中:
\begin{itemize}
  \item \(T_{\text{total}}\):任务总完成时延。若为本地执行,则为本地计算时延;若为卸载,则为通信时延与远程计算时延之和,远程计算时延由并发开销模型精确预测;
  \item \(E_{\text{cost}}\):任务执行的能量消耗,本地执行主要消耗计算能量,卸载执行主要消耗通信能量;
  \item \(B_{\text{imbalance}}\):网络负载不均衡度,可通过所有 SES 负载的方差衡量,选择空闲节点有助于降低该值;
  \item \(\omega_T, \omega_E, \omega_B\) 为对应权重系数,用于按业务需求(如时延优先、节能优先)调整优化侧重点。
\end{itemize}

\paragraph{(5)策略(Policy, \(\pi\))}

策略 \(\pi(a|s)\) 定义了在给定状态 \(s\) 下选择动作 \(a\) 的概率。DRL 的目标是学习最优策略 \(\pi^*\),最大化长期折扣奖励的期望:
\[
\pi^* = \arg\max_{\pi} \mathbb{E}\left[\sum_{t=0}^{\infty} \gamma^{t} r_t \mid s_0, \pi\right],
\]
其中 \(\gamma \in [0,1)\) 为折扣因子。

\subsection{基于深度 Q 网络(DQN)的求解算法}

对于上述离散动作空间的 MDP 问题,深度 Q 网络(Deep Q-Network, DQN)是一种经典且有效的求解算法[10]。DQN 用深度神经网络近似最优动作价值函数 \(Q^*(s,a)\):
\[
Q^*(s,a) \approx Q(s,a;\theta),
\]
其中 \(\theta\) 为神经网络参数。网络输入为状态 \(s\),输出为每个可能动作的 Q 值。在给定状态下,智能体选择 Q 值最大的动作:
\[
a = \arg\max_{a' \in A} Q(s,a';\theta).
\]

训练 DQN 时,本文借鉴前期工作[20] 并结合天基网络特点,采用以下关键技术:

\begin{itemize}
  \item \textbf{经验回放(Experience Replay):}将与环境交互产生的经验元组 \((s_t,a_t,r_t,s_{t+1})\) 存入经验池,训练时从中随机采样小批量数据,打破时序关联,提升训练稳定性。
  \item \textbf{目标网络(Target Network):}采用在线网络 \(Q(s,a;\theta)\) 与目标网络 \(Q(s,a;\theta^-)\) 双网络结构,目标网络参数以较低频率从在线网络复制,有助于稳定训练目标。
  \item \textbf{损失函数:}以满足贝尔曼方程为目标,最小化均方误差:
  \[
    L(\theta) = \mathbb{E}_{(s,a,r,s')}\!\left[\big(r + \gamma \max_{a'} Q(s',a';\theta^-) - Q(s,a;\theta)\big)^2\right].
  \]
\end{itemize}

在天基网络场景中,可引入文献[12] 中的时空注意力机制增强 Q 网络对动态状态的表征能力。对于更复杂的协同场景,还可扩展为多智能体强化学习(Multi-Agent RL, MARL)框架[13, 15],让多个 ST 智能体共同学习协作策略。

通过 DRL 赋能,每个卫星终端均可成为具备自主学习与智能决策能力的实体,能够在不断变化的环境中动态权衡时延、能耗与网络负载等多重目标,为每个任务选择最优执行路径,从而在系统层面实现资源的自适应、高效调度与优化。

\section{天基网络计算卸载的形式化建模}

在前述章节对系统架构与关键技术进行阐述的基础上,本节对天基网络计算卸载问题进行统一、形式化的数学建模。该模型整合服务时延、并发开销、能耗等关键性能指标,并明确本研究旨在求解的优化目标。

\subsection{任务模型}

假设在时间 \(t\) 有一个计算任务 \(i\) 由卫星终端 \(\text{ST}_i\) 生成,该任务可表示为元组:
\[
  \text{Task}_i = \{C_i, D_i, T_i^{\max}\},
\]
其中:
\begin{itemize}
  \item \(C_i\):任务 \(i\) 的计算复杂度,以所需 CPU 周期数或 FLOPs 表示;
  \item \(D_i\):任务 \(i\) 的数据大小(比特),主要考虑输入数据;
  \item \(T_i^{\max}\):任务 \(i\) 的最大可容忍时延(截止时间)。
\end{itemize}

\subsection{决策变量}

对于每个任务 \(i\),其卸载决策由向量 \(X_i = \{x_{i,j}\}\) 表示,其中 \(j \in \{0,1,\dots,M\}\),\(M\) 为可用 SES 节点数量:
\[
  x_{i,j} \in \{0,1\}, \quad \sum_{j=0}^{M} x_{i,j} = 1.
\]
其中:
\begin{itemize}
  \item \(x_{i,0} = 1\) 表示任务 \(i\) 在本地 \(\text{ST}_i\) 上执行;
  \item \(x_{i,j} = 1\)(\(j>0\))表示任务 \(i\) 被卸载至第 \(j\) 个 SES 执行。
\end{itemize}

\subsection{时延模型(Latency Model)}

任务 \(i\) 的总完成时延 \(T_i\) 随卸载决策不同而变化。

\paragraph{(1)本地执行时延}

若任务本地执行(\(x_{i,0}=1\)),总时延为
\[
  T_i^{\text{local}} = \frac{C_i}{f_{\text{local}}},
\]
其中 \(f_{\text{local}}\) 为 \(\text{ST}_i\) 的本地计算能力。

\paragraph{(2)卸载执行时延}

若任务被卸载至 SES\(_j\)(\(x_{i,j}=1, j>0\)),其总时延为
\[
  T_i^{\text{offload},j} = T_i^{\text{up},j} + T_i^{\text{comp},j} + T_i^{\text{down},j},
\]
其中:
\begin{itemize}
  \item 上行传输时延:
  \[
    R_{i,j} = B_{i,j} \log_2\!\left(1 + \frac{P_i H_{i,j}}{N_0}\right), \quad
    T_i^{\text{up},j} = \frac{D_i}{R_{i,j}},
  \]
  其中 \(B_{i,j}\) 为带宽,\(P_i\) 为发射功率,\(H_{i,j}\) 为信道增益,\(N_0\) 为噪声功率。下行结果时延 \(T_i^{\text{down},j}\) 通常可视为常数或近似忽略;
  \item 远程计算时延:
  \[
    T_i^{\text{comp},j} = T_i^{\text{queue},j} + T_i^{\text{exec},j},
  \]
  其中排队时延 \(T_i^{\text{queue},j}\) 取决于当前队列状态,执行时延 \(T_i^{\text{exec},j}\) 由并发开销模型 \(f_{\text{co}}\) 预测:
  \[
    T_i^{\text{exec},j} = f_{\text{co}}\!\left(j, C_i, \text{TaskSet}_j\right),
  \]
  \(\text{TaskSet}_j\) 为可从区块链账本推断的当前在 SES\(_j\) 上执行的任务集合。
\end{itemize}

因此,任务 \(i\) 的总时延统一表示为
\[
  T_i = x_{i,0} T_i^{\text{local}} + \sum_{j=1}^{M} x_{i,j} T_i^{\text{offload},j}.
\]

\subsection{能耗模型(Energy Model)}

任务 \(i\) 的总能耗 \(E_i\) 亦随卸载决策变化,本文主要关注 ST 侧能耗。

\paragraph{(1)本地执行能耗}
\[
  E_i^{\text{local}} = \kappa C_i,
\]
其中 \(\kappa\) 为每 CPU 周期能耗系数。

\paragraph{(2)卸载执行能耗}

主要为上行通信能耗:
\[
  E_i^{\text{offload},j} = P_i T_i^{\text{up},j} = P_i \frac{D_i}{R_{i,j}}.
\]

故总能耗可写为
\[
  E_i = x_{i,0} E_i^{\text{local}} + \sum_{j=1}^{M} x_{i,j} E_i^{\text{offload},j}.
\]

\subsection{优化目标}

本文的最终目标是学习一个卸载策略 \(\pi\),对于一系列到达的任务 \(i=1,2,\dots,N\),最小化综合时延、能耗与负载均衡成本的长期平均:
\[
  \min_{\pi} \lim_{N \to \infty} \frac{1}{N} \sum_{i=1}^{N}
  \left(\omega_T T_i + \omega_E E_i + \omega_B B_t\right),
\]
其中 \(B_t\) 为时刻 \(t\) 的网络负载不均衡度,可定义为所有 SES 归一化负载的方差:
\[
  B_t = \text{Var}\!\left(\frac{\text{Load}_1(t)}{f_1^{\max}}, \dots,
  \frac{\text{Load}_M(t)}{f_M^{\max}}\right),
\]
\(\text{Load}_j(t)\) 为节点 \(j\) 在时刻 \(t\) 的计算负载,可由区块链账本推断,\(\omega_T,\omega_E,\omega_B\) 为各目标权重。

约束条件包括:
\begin{itemize}
  \item 卸载决策约束:\(\forall i,j:\ x_{i,j} \in \{0,1\}, \sum_{j=0}^{M} x_{i,j} = 1\);
  \item 时延约束:\(\forall i:\ T_i \leq T_i^{\max}\);
  \item 能量约束:\(\forall i:\ E_i \leq E_{\text{budget}}\);
  \item SES 容量约束:任意时刻 SES\(_j\) 的总计算资源分配不超过 \(f_j^{\max}\)。
\end{itemize}

将上述多目标优化问题对应到第~2.4 节中的 DRL 奖励函数,即可通过 DRL 算法进行求解,在满足约束的前提下寻找长期最大化系统效用(或最小化成本)的动态卸载策略。

\section{本章小结}

本章围绕天基网络中卫星终端试验任务的计算卸载问题,构建了完整的理论框架与技术体系。首先,定义了面向的三层系统架构,并深入分析了网络拓扑高动态性、资源状态信息不完全、任务并发性能影响以及分布式协同信任缺失等四大核心挑战。

随后,本文依次阐述了拟采用的关键技术路径:一是基于分层联盟链的分布式可信管理技术,通过主子链架构与链上链下协同存储机制以及智能合约,实现试验任务全生命周期的防篡改存证与非实时全局状态共享;二是基于联邦学习的并发开销预测技术,在保护节点数据隐私的前提下协同训练高精度预测模型,为卸载决策提供精细化成本先验;三是基于深度强化学习的动态调度技术,将计算卸载建模为 MDP,并设计综合考虑时延、能耗和负载均衡的多目标奖励函数,使智能体在高度动态环境中自适应学习近似最优策略。

最后,本章对天基网络计算卸载问题进行了统一的形式化建模,清晰给出任务模型、决策变量、时延与能耗计算方式以及系统优化目标,形成“区块链提供状态 \(\rightarrow\) FL 预测成本 \(\rightarrow\) DRL 优化决策”的技术闭环,为后续章节的算法设计与仿真实验提供了坚实理论基础。
