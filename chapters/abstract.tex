%%
% BIThesis 研究生学位论文模板 The BIThesis Template for Graduate Thesis
% This file has no copyright assigned and is placed in the Public Domain.
%%
\begin{abstract} 近年来,随着天地一体化信息网络与“新基建”战略的深入推进,天基网络正经历从单一通信传输向边缘智能计算的跨越式转变。卫星终端作为天基数据的感知与处理前哨,在偏远地区监测、应急响应及各类在轨试验中发挥着不可替代的作用。然而,面对日益复杂的计算密集型试验任务,受限于星载设备的体积、功耗及抗辐射设计,卫星终端的计算资源极其有限。当海量任务并发产生时,传统的计算卸载方案往往忽视了资源竞争带来的“并发开销”,且依赖地面集中式决策导致的高时延与信令开销,已难以满足天基试验任务的实时性与可靠性需求。

  针对上述挑战,本文提出了一种融合区块链、联邦学习(Federated Learning, FL)与深度强化学习(DRL)的卫星终端智能管控与计算卸载架构。该架构旨在利用联邦学习打破数据孤岛,在保护隐私的前提下实现协作式环境感知;利用区块链构建可信协作底座,解决分布式环境下的信任与状态同步问题。本文的主要研究内容与创新点如下:
  
  (1)基于区块链的卫星试验任务可信存证与协同管理:针对多源异构试验数据的安全与一致性问题,设计了基于联盟链的分层式数据管理架构。利用智能合约实现试验任务全生命周期的自动化记录与防篡改存证,并构建非实时全局状态共享机制,为分布式协同计算提供可信的数据底座,确保了任务执行过程的可追溯性与多方协作的安全性。
  
  (2)基于联邦学习的并发开销预测与卸载决策优化:针对多任务并发场景下的资源竞争难题,提出了一种基于FL与深度Q网络(DQN)的分布式卸载策略。通过在卫星终端间开展联邦协同训练,构建高精度的任务并发开销预测模型,使终端能够在不共享原始任务数据的前提下感知全局负载压力,从而实现任务卸载决策的智能化与隐私保护,有效解决了资源竞争导致的时延剧增问题。
  
  (3)基于深度强化学习的动态资源调度与节点优选:针对天基网络拓扑高动态变化及星间链路不稳定的特性,设计了基于改进PPO/DQN算法的动态资源调度模型。该模型将时延、能耗、负载均衡度纳入多目标优化函数,通过智能体与环境的持续交互,实现复杂动态约束下的计算资源自适应分配与最优卸载节点选择。
  
  (4)天基分布式试验智能管控平台的设计与验证:基于上述理论成果,设计并研制了包含卫星终端模拟、区块链网络、智能调度引擎及可视化监控的一体化演示验证平台。在典型的卫星试验任务场景下进行了系统级测试,实验结果表明,本文提出的方案在降低平均任务时延、提升系统吞吐量及保障数据隐私方面均优于传统基准算法,验证了所提架构的有效性与优越性。
  
  本研究探索了区块链与人工智能技术在天基网络中的深度融合应用,有效提升了受限资源下卫星终端的任务处理效能与数据安全水平,为未来构建高效、智能、可信的天基边缘计算网络提供了重要的理论依据与技术支撑。
\end{abstract}

% 如需手动控制换行连字符位置,可写 aa\-bb,详见
% https://bithesis.bitnp.net/faq/hyphen.html

\begin{abstractEn}
  In recent years, with the in-depth advancement of space-ground integrated information networks and "New Infrastructure" strategies, space-based networks are undergoing a leapfrog transition from simple communication transmission to edge intelligent computing. As the forward outpost for space-based data perception and processing, satellite terminals play an irreplaceable role in remote area monitoring, emergency response, and various on-orbit experiments. However, faced with increasingly complex computation-intensive test tasks, satellite terminals are severely constrained by the size, weight, and power (SWaP) limitations of onboard equipment. When massive tasks are generated concurrently, traditional computation offloading schemes often overlook the "concurrency overhead" caused by resource competition. Furthermore, reliance on ground-based centralized decision-making leads to high latency and signaling overhead, making it difficult to meet the real-time and reliability requirements of space-based test tasks.

  To address these challenges, this thesis proposes an intelligent management and computation offloading architecture for satellite terminals, fusing Blockchain, Federated Learning (FL), and Deep Reinforcement Learning (DRL). This architecture aims to utilize FL to break data silos and achieve collaborative environmental perception while preserving privacy; simultaneously, it leverages blockchain to build a trusted collaboration foundation, resolving trust and state synchronization issues in distributed environments. The main research contents and innovations are as follows:
  
  (1) Trusted Storage and Collaborative Management of Satellite Test Tasks Based on Blockchain: Addressing the security and consistency issues of multi-source heterogeneous test data, a hierarchical data management architecture based on consortium blockchain is designed. Smart contracts are utilized to realize automated recording and tamper-proof storage of the entire lifecycle of test tasks. Additionally, a non-real-time global state sharing mechanism is constructed to provide a trusted data foundation for distributed collaborative computing, ensuring the traceability of the execution process and the security of multi-party collaboration.
  
  (2) Concurrency Overhead Prediction and Offloading Decision Optimization Based on Federated Learning: Targeting the resource competition problem in multi-task concurrency scenarios, a distributed offloading strategy based on FL and Deep Q-Network (DQN) is proposed. By conducting federated collaborative training among satellite terminals, a high-precision prediction model for task concurrency overhead is constructed. This enables terminals to perceive global load pressure without sharing raw task data, thereby achieving intelligent and privacy-preserving offloading decisions and effectively solving the latency surge caused by resource competition.
  
  (3) Dynamic Resource Scheduling and Node Selection Based on Deep Reinforcement Learning: Addressing the highly dynamic topology of space-based networks and the instability of inter-satellite links, a dynamic resource scheduling model based on improved PPO/DQN algorithms is designed. This model incorporates latency, energy consumption, and load balance into a multi-objective optimization function. Through continuous interaction between the agent and the environment, adaptive allocation of computing resources and optimal offloading node selection under complex dynamic constraints are achieved.
  
  (4) Design and Verification of Space-based Distributed Test Intelligent Management Platform: Based on the above theoretical results, an integrated demonstration and verification platform containing satellite terminal simulation, blockchain network, intelligent scheduling engine, and visual monitoring is designed and developed. System-level tests under typical satellite test task scenarios demonstrate that the proposed scheme outperforms traditional benchmark algorithms in reducing average task latency, improving system throughput, and guaranteeing data privacy, verifying the effectiveness and superiority of the proposed architecture.
  
  This research explores the deep integration and application of blockchain and artificial intelligence technologies in space-based networks, effectively improving the task processing efficiency and data security level of satellite terminals under resource-constrained conditions, providing important theoretical basis and technical support for building efficient, intelligent, and trusted space-based edge computing networks in the future.\end{abstractEn}
